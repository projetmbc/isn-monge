\subsection{Images numériques}

% ------------- %

\frame{
	\frametitle{Numériser, c'est coder !}

	Quel codage choisir qui réponde aux contraintes technologiques ?
		
	\begin{itemize}[<+(1)->]
		\item Images représentées.
		
		\begin{itemize}[<+(1)->]
			\item Quels émetteurs rouges, verts et bleus activés ?

			\item Quelle intensité pour un émetteur donné ?
		\end{itemize}
		
		\item Images captées.
		
		\begin{itemize}[<+(1)->]
			\item Quantifier les mesures physiques.
		\end{itemize}
			
		\item Nécessité de définir des conventions.
	\end{itemize}
}


% ------------- %

\frame{
	\frametitle{Coder les couleurs}

	Le modèle RVB (Rouge-Vert-Bleu) ou RGB (Red-Green-Blue).
	
	\begin{itemize}[<+(1)->]
		\item Souhait de modéliser les outils physiques utilisés.
		
		\item Pour chaque couleur : valeurs allant de $0$ (intensité nulle) 
		
		à $255$ (intensité maximale).
		
		\item Combien de chiffres nécessaires pour un codage binaire ?
		
		\pause 
		
		$256 = 2^8$ donc besoin d'au moins $3 \times 8 = 24$ 
		chiffres en écriture binaire.
	\end{itemize}
}


% ------------- %

\frame{
	\frametitle{Coder les couleurs}

	Exemples de couleur RGB avec la notation
	(Red , Green , Blue).
	
	\begin{itemize}[<+(1)->]
		\item Quelle couleur correspond à $(255 , 0 , 0)$ ?
		
		\pause
		Avec \LaTeX, on obtient
		\framebox{\textcolor[RGB]{255 , 0 , 0}{$\blacksquare\!\blacksquare\!\blacksquare$}} 
		qui est un rouge.
		
		\item Quelle couleur correspond à $(0 , 255 , 0)$ ?
		
		\pause
		Avec \LaTeX, on obtient
		\framebox{\textcolor[RGB]{0 , 255 , 0}{$\blacksquare\!\blacksquare\!\blacksquare$}} 
		qui est un vert.
		
		\item Quelle couleur correspond à $(0 , 0, 255)$ ?
		
		\pause
		Avec \LaTeX, on obtient
		\framebox{\textcolor[RGB]{0 , 0, 255}{$\blacksquare\!\blacksquare\!\blacksquare$}} 
		qui est un bleu.
		
		\item Quel type de couleur correspond à $(200 , 200, 200)$ ?
		
		\pause
		Avec \LaTeX, on obtient
		\framebox{\textcolor[RGB]{200, 200, 200}{$\blacksquare\!\blacksquare\!\blacksquare$}} 
		qui est un gris.
		
		Plus généralement, $256$ nuances de gris grâce à
		$(n , n , n)$. 
	\end{itemize}
}


% ------------- %

\frame{
	\frametitle{Coder les couleurs}

	Exemples de couleur RGB avec la notation
	(Red , Green , Blue).
	
	\vspace{-.7em}
	\begin{flushright}
		\scriptsize (la suite)
	\end{flushright}
	\vspace{-1.2em}
	
	\begin{itemize}[<+(1)->]		
		\item Comment obtenir du blanc ?
		
		\pause
		$(255 , 255, 255)$ nous donne avec \LaTeX : \framebox{\textcolor[RGB]{255,255,255}{$\blacksquare\!\blacksquare\!\blacksquare$}} .
		
		\item Comment obtenir du noir ?
		
		\pause
		$(0 , 0, 0)$ nous donne avec \LaTeX : \framebox{\textcolor[RGB]{0,0,0}{$\blacksquare\!\blacksquare\!\blacksquare$}} .
		
		\item Proposons un codage correspondant à un jaune.
		
		\pause
		Mélangeons du rouge et du vert (pas simple a priori). 
		
		$(255 , 255, 0)$ nous donne avec \LaTeX : \framebox{\textcolor[RGB]{255,255,0}{$\blacksquare\!\blacksquare\!\blacksquare$}} .
	\end{itemize}
}


% ------------- %

\frame{
	\frametitle{Coder les couleurs}

	Une autre convention pour le format RGB.
	
	\pause
	\medskip
	
	$256 = 2^8 = (2^4)^2 = 16^2$ donc deux chiffres en écriture hexadécimale pour chacune des couleurs.
	
	\pause
	\medskip
	
	Collage successif de ces écritures dans l'ordre R-G-B.
	
	\medskip
	
	$3 \times 2 = 6$ chiffres au total pour les trois couleurs.
	
	\pause
	\medskip
	
	Par exemple, $(0 , 141, 255)$ s'écrit
	\pause
	008DFF ou juste 8DFF
	en utilisant $141 = 128 + 13 = 8 \times 16 + 13$.
	
	\pause
	\medskip
	
	Un autre exemple : (30 , 173, 5) s'écrit
	\pause
	1EAD05.
}


% ------------- %

\frame{
	\frametitle{Repérage des émetteurs et des récepteurs}

	Reste à indiquer où une couleur sera activée, ou bien à savoir 
	
	où elle a été mesurée (quantifiée).
	
	\pause
	\medskip
	
	\textbf{Attention !} Convention des écritures manuscrites  \og{}latines\fg{}
	
	ou de la machine à écrire.
	
	\begin{center}
		\definecolor{qqffqq}{rgb}{0.,1.,0.}
\definecolor{zzttqq}{rgb}{0.6,0.2,0.}
\definecolor{ffqqqq}{rgb}{1.,0.,0.}
\begin{tikzpicture}[line cap=round,line join=round,>=triangle 45,x=0.6cm,y=0.6cm]
\clip(-3.840937894924428,-1.30338099546581) rectangle (6.243481310215926,5.683680882381437);
\fill[color=ffqqqq,fill=ffqqqq,fill opacity=1.0] (0.,4.) -- (1.,4.) -- (1.,3.) -- (0.,3.) -- cycle;
\fill[color=qqffqq,fill=qqffqq,fill opacity=1.0] (2.,2.) -- (3.,2.) -- (3.,1.) -- (2.,1.) -- cycle;
\fill[fill=black,fill opacity=1.0] (-2.,2.) -- (1.,2.) -- (1.,1.) -- (-2.,1.) -- cycle;
\draw (-3.,5.)-- (4.,5.);
\draw (4.,5.)-- (4.,1.);
\draw (4.,1.)-- (-3.,1.);
\draw (-3.,1.)-- (-3.,5.);
\draw (-2.,5.)-- (-2.,1.);
\draw (-1.,1.)-- (-1.,5.);
\draw (0.,5.)-- (0.,1.);
\draw (1.,1.)-- (1.,5.);
\draw (2.,5.)-- (2.,1.);
\draw (3.,1.)-- (3.,5.);
\draw (4.,4.)-- (-3.,4.);
\draw (-3.,3.)-- (4.,3.);
\draw (4.,2.)-- (-3.,2.);
\draw [->] (-3.,5.) -- (-3.,-1.);
\draw [->] (-3.,5.) -- (6.,5.);
\draw (5.681635097358106,5.770118761282641) node[anchor=north west] {$\mathbf{x}$};
\draw (-3.696874763422423,-0.5254400853549817) node[anchor=north west] {$\mathbf{y}$};
\draw [color=zzttqq] (0.,4.)-- (1.,4.);
\draw [color=zzttqq] (1.,4.)-- (1.,3.);
\draw [color=zzttqq] (1.,3.)-- (0.,3.);
\draw [color=zzttqq] (0.,3.)-- (0.,4.);
\draw [color=zzttqq] (2.,2.)-- (3.,2.);
\draw [color=zzttqq] (3.,2.)-- (3.,1.);
\draw [color=zzttqq] (3.,1.)-- (2.,1.);
\draw [color=zzttqq] (2.,1.)-- (2.,2.);
\draw (-2.,2.)-- (1.,2.);
\draw (1.,2.)-- (1.,1.);
\draw (1.,1.)-- (-2.,1.);
\draw (-2.,1.)-- (-2.,2.);
\draw (-2.803683348109992,5.770118761282641) node[anchor=north west] {$\mathbf{0}$};
\draw (-1.7952414275959563,5.770118761282641) node[anchor=north west] {$\mathbf{1}$};
\draw (-0.8012058202321215,5.770118761282641) node[anchor=north west] {$\mathbf{2}$};
\draw (0.19282978713171336,5.770118761282641) node[anchor=north west] {$\mathbf{3}$};
\draw (1.2012717076457486,5.770118761282641) node[anchor=north west] {$\mathbf{4}$};
\draw (2.1953073150095834,5.770118761282641) node[anchor=north west] {$\mathbf{5}$};
\draw (3.203749235523619,5.770118761282641) node[anchor=north west] {$\mathbf{6}$};
\draw (-3.696874763422423,4.948958911721212) node[anchor=north west] {$\mathbf{0}$};
\draw (-3.696874763422423,3.9549233043573766) node[anchor=north west] {$\mathbf{1}$};
\draw (-3.696874763422423,2.946481383843341) node[anchor=north west] {$\mathbf{2}$};
\draw (-3.696874763422423,1.9524457764795058) node[anchor=north west] {$\mathbf{3}$};
\end{tikzpicture}

		
		\vspace{-0.5em}
		{\scriptsize Graphique produit avec GeoGebra}
	\end{center}
}


% ------------- %

\frame{
	\frametitle{Repérage des émetteurs et des récepteurs}
	
	\begin{center}
		\definecolor{qqffqq}{rgb}{0.,1.,0.}
\definecolor{zzttqq}{rgb}{0.6,0.2,0.}
\definecolor{ffqqqq}{rgb}{1.,0.,0.}
\begin{tikzpicture}[line cap=round,line join=round,>=triangle 45,x=0.6cm,y=0.6cm]
\clip(-3.840937894924428,-1.30338099546581) rectangle (6.243481310215926,5.683680882381437);
\fill[color=ffqqqq,fill=ffqqqq,fill opacity=1.0] (0.,4.) -- (1.,4.) -- (1.,3.) -- (0.,3.) -- cycle;
\fill[color=qqffqq,fill=qqffqq,fill opacity=1.0] (2.,2.) -- (3.,2.) -- (3.,1.) -- (2.,1.) -- cycle;
\fill[fill=black,fill opacity=1.0] (-2.,2.) -- (1.,2.) -- (1.,1.) -- (-2.,1.) -- cycle;
\draw (-3.,5.)-- (4.,5.);
\draw (4.,5.)-- (4.,1.);
\draw (4.,1.)-- (-3.,1.);
\draw (-3.,1.)-- (-3.,5.);
\draw (-2.,5.)-- (-2.,1.);
\draw (-1.,1.)-- (-1.,5.);
\draw (0.,5.)-- (0.,1.);
\draw (1.,1.)-- (1.,5.);
\draw (2.,5.)-- (2.,1.);
\draw (3.,1.)-- (3.,5.);
\draw (4.,4.)-- (-3.,4.);
\draw (-3.,3.)-- (4.,3.);
\draw (4.,2.)-- (-3.,2.);
\draw [->] (-3.,5.) -- (-3.,-1.);
\draw [->] (-3.,5.) -- (6.,5.);
\draw (5.681635097358106,5.770118761282641) node[anchor=north west] {$\mathbf{x}$};
\draw (-3.696874763422423,-0.5254400853549817) node[anchor=north west] {$\mathbf{y}$};
\draw [color=zzttqq] (0.,4.)-- (1.,4.);
\draw [color=zzttqq] (1.,4.)-- (1.,3.);
\draw [color=zzttqq] (1.,3.)-- (0.,3.);
\draw [color=zzttqq] (0.,3.)-- (0.,4.);
\draw [color=zzttqq] (2.,2.)-- (3.,2.);
\draw [color=zzttqq] (3.,2.)-- (3.,1.);
\draw [color=zzttqq] (3.,1.)-- (2.,1.);
\draw [color=zzttqq] (2.,1.)-- (2.,2.);
\draw (-2.,2.)-- (1.,2.);
\draw (1.,2.)-- (1.,1.);
\draw (1.,1.)-- (-2.,1.);
\draw (-2.,1.)-- (-2.,2.);
\draw (-2.803683348109992,5.770118761282641) node[anchor=north west] {$\mathbf{0}$};
\draw (-1.7952414275959563,5.770118761282641) node[anchor=north west] {$\mathbf{1}$};
\draw (-0.8012058202321215,5.770118761282641) node[anchor=north west] {$\mathbf{2}$};
\draw (0.19282978713171336,5.770118761282641) node[anchor=north west] {$\mathbf{3}$};
\draw (1.2012717076457486,5.770118761282641) node[anchor=north west] {$\mathbf{4}$};
\draw (2.1953073150095834,5.770118761282641) node[anchor=north west] {$\mathbf{5}$};
\draw (3.203749235523619,5.770118761282641) node[anchor=north west] {$\mathbf{6}$};
\draw (-3.696874763422423,4.948958911721212) node[anchor=north west] {$\mathbf{0}$};
\draw (-3.696874763422423,3.9549233043573766) node[anchor=north west] {$\mathbf{1}$};
\draw (-3.696874763422423,2.946481383843341) node[anchor=north west] {$\mathbf{2}$};
\draw (-3.696874763422423,1.9524457764795058) node[anchor=north west] {$\mathbf{3}$};
\end{tikzpicture}

	\end{center}

	\vspace{-0.5em}
	
	Cellule \textcolor[RGB]{255,0,0}{rouge} en 
	\pause
	$(x , y) = (3, 1)$.
	
	\medskip
	
	Cellule \textcolor[RGB]{0,255,0}{verte} en 
	\pause
	$(x , y) = (5, 3)$.
	
	\medskip
	
	Couleur des cellules en $(1, 3)$ , $(2, 3)$ et $(3, 3)$ :
	\pause
	le noir. 
}
