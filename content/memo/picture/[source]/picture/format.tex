\subsection{Formats classiques d'images}

% ------------- %

\frame{
	\frametitle{Formats \texttt{ASCII}}

	Caractéristiques générales.

	\begin{itemize}[<+(1)->]		
		\item Formats \texttt{ASCII} \og{}manipulables à la main\fg{}.

		\begin{itemize}	
			\item Windows : utilisation de Notepad++.
			
			\item Linux : Geany.
			
			\item Mac OS : Atom.
		\end{itemize}		
		
		\item Formats pris en charge par Gimp.
		
		\item Trois versions.

		\begin{itemize}	
			\item Portable Bitmap (PBM) : image en noir et blanc.
			
			\item Portable Graymap (PGM) : image en niveau de gris.
			
			\item Portable Pixmap (PPM) : image en couleur.
		\end{itemize}
		
		\item Pour ceux présentés, pas de compression : les infos de chaque pixel directement dans le fichier image.
	\end{itemize}		
}


% ------------- %

\frame{
	\frametitle{Formats \texttt{ASCII} : Portable Bitmap (PBM)}

	Spécifications techniques du fichier.

	\begin{itemize}[<+(1)->]		
		\item Extension du fichier : \texttt{.pbm}.
		
		\item 1ère ligne ne contenant que le texte \texttt{P1}.
		
		\item 2ème ligne : dimensions sous la forme 
		\texttt{LARGEUR HAUTEUR}.
		
		\item À partir de la 3ème ligne.

		\begin{itemize}	
			\item \texttt{1} code le noir et \texttt{0} le blanc.
			\item Espaces possibles mais pas indispensables.
			\item Pas plus de 70 caractères par ligne.
		\end{itemize}
		
		\item Des commentaires : texte précédé par le symbole \texttt{\#}.
	\end{itemize}		
}


% ------------- %

\begin{frame}[fragile]
	\frametitle{Formats \texttt{ASCII} : Portable Bitmap (PBM)}

	Un exemple.

	\begin{multicols}{2}
		\scriptsize
		\begin{verbatim}
P1     # Image en noir et blanc
13 11  # Largeur = 13  Hauteur = 11
0000000000000
0000000000000
0001111100000
0001000100000
0001011111000
0001010101000
0001111101000
0000010001000
0000011111000
0000000000000
0000000000000
		\end{verbatim}
		
		\vfill
		\columnbreak
		
		\begin{center}
			\framebox{\includegraphics[width=0.5\textwidth]{picture/carres.png}}
			
			\medskip
			
			\textbf{ATTENTION !} 
			
			Image encadrée et agrandie.
		\end{center}
	\end{multicols}		
\end{frame}


% ------------- %

\frame{
	\frametitle{Formats \texttt{ASCII} : Portable Graymap (PGM)}

	Spécifications techniques du fichier.

	\begin{itemize}[<+(1)->]		
		\item Extension du fichier : \texttt{.pgm}.
		
		\item 1ère ligne ne contenant que le texte \texttt{P2}.
		
		\item 2ème ligne : dimensions sous la forme 
		\texttt{LARGEUR HAUTEUR}.
		
		\item 3ème ligne : \texttt{MAX} la valeur maximale de niveau de gris.
		
		On doit avoir \texttt{MAX} $\leqslant$ 65 536.
		
		\item À partir de la 4ème ligne.

		\begin{itemize}	
			\item Gris codé de \texttt{0} pour le noir à \texttt{MAX} pour le blanc.
			\item Espaces indispensables.
			\item Pas plus de 70 caractères par ligne.
		\end{itemize}
		
		\item Des commentaires : texte précédé par le symbole \texttt{\#}.
	\end{itemize}		
}


% ------------- %

\begin{frame}[fragile]
	\frametitle{Formats \texttt{ASCII} : Portable Graymap (PGM)}

	Un exemple.

	\begin{multicols}{2}
		\scriptsize
		\begin{verbatim}
P2   # Image en niveau de gris
8 7  # Largeur = 8  Hauteur = 7
10   # 11 Niveaux de gris
10 10 10 10 10 10 10 10
10 0  0  0  0  0  10 10
10 0  10 10 10 0  5  10
10 0  10 10 10 0  5  10
10 0  10 10 10 0  5  10
10 0  0  0  0  0  5  10
10 10 5  5  5  5  5  10
10 10 10 10 10 10 10 10
		\end{verbatim}
		
		\vfill
		\columnbreak
		
		\vspace{-0.5em}
		
		\begin{center}
			\framebox{\includegraphics[width=0.5\textwidth]{picture/carre_ombre.png}}
			
			\medskip
			
			\textbf{ATTENTION !} 
			
			Image encadrée et agrandie.
		\end{center}
	\end{multicols}		
\end{frame}


% ------------- %

\frame{
	\frametitle{Formats \texttt{ASCII} : Portable Pixmap (PPM)}

	Spécifications techniques du fichier.

	\begin{itemize}[<+(1)->]		
		\item Extension du fichier : \texttt{.ppm}.
		
		\item 1ère ligne ne contenant que le texte \texttt{P3}.
		
		\item 2ème ligne : dimensions sous la forme 
		\texttt{LARGEUR HAUTEUR}.
		
		\item 3ème ligne : \texttt{MAX} la valeur maximale pour le rouge, le vert et le bleu. On doit avoir \texttt{MAX} $\leqslant$ 65 536.
		
		\item À partir de la 4ème ligne.

		\begin{itemize}	
			\item Couleur codé au format R G B.
			\item Espaces indispensables.
			\item Pas plus de 70 caractères par ligne.
		\end{itemize}
		
		\item Des commentaires : texte précédé par le symbole \texttt{\#}.
	\end{itemize}		
}


% ------------- %

\begin{frame}[fragile]
	\frametitle{Formats \texttt{ASCII} : Portable Pixmap (PPM)}

	Un exemple.

	\begin{center}
		\scriptsize
		\begin{verbatim}
P3   # Image en couleur
7 2  # Largeur = 7 Hauteur = 2
10   # 11 niveaux pour R, V ou B
0 0 0   5 0 0   0 0 0   0 5 0   0 0 0   8 8 8   0 0 0
0 0 0   5 0 0   0 0 0   0 5 0   0 0 0   8 8 8   0 0 0
		\end{verbatim}
		
		\bigskip
		
		\framebox{\includegraphics[width=0.5\textwidth]{picture/trois_traits.png}}
		
		\medskip
			
		\textbf{ATTENTION !} 
			
		Image encadrée et agrandie.
	\end{center}		
\end{frame}


% ------------- %

\frame{
	\frametitle{Formats utiles pour le web}

	\og{}Non manipulables à la main\fg{} :
	utilisation de pillow (Python 3).

	\begin{itemize}[<+(1)->]		
		\item Graphics Interchange Format (GIF)
		
		\begin{itemize}		
			\item Extension du fichier : \texttt{.gif}.
			\item Transparence et compression (avec perte).
			\item Affichage progressif.
		\end{itemize}
	
		\item Portable Network Graphics (PNG) 
		
		\begin{itemize}		
			\item Extension du fichier : \texttt{.png}.
			\item Transparence et compression sans perte.
			\item Affichage progressif.
		\end{itemize}
	
		\item Joint Photographic Experts Group (JPEG)
		
		\begin{itemize}		
			\item Extension du fichier : \texttt{.jpg} ou \texttt{.jpeg}.
			\item Pas de transparence mais compression réglable  (avec perte).
			\item Affichage progressif.
		\end{itemize}
	\end{itemize}		
}
