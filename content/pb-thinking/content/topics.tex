\begin{description}
	\item[Thème 1 : Persistance de l'information.] 
	Les données, notamment personnelles, sont susceptibles d'être mémorisées pour de longues périodes sans maîtrise par les personnes concernées.

	Prendre conscience de la persistance de l'information sur les espaces numériques interconnectés.

	Comprendre les principes généraux permettant de se comporter de façon responsable par rapport au droit des personnes dans les espaces numériques.

	La persistance de l'information se manifeste tout particulièrement au sein des disques durs mais aussi des mémoires caches. Elle interagit avec le droit à la vie privée et fait naître une revendication du "droit à l'oubli".


	\item[Thème 2 : Supranationalité des réseaux.] 
	Prendre conscience du caractère supranational des réseaux et des conséquences sociales, économiques et politiques qui en découlent.

	On met en évidence le fait que certains pays autorisent la mise en ligne d'informations, services ou contenus numériques dont la consultation n'est pas permise dans d'autres pays.


	\item[Thème 3 : Les licences logiciels et les droits afférents] 
	Qu'est ce qu'un logiciel libre ? libre / gratuit / opensource.

	Les modèles économiques.

	Licences GNU, Creative Commons...


	\item[Thème 4 : RGPD et CNIL] 
	Qu'est ce que la CNIL ? Fichage et libertés individuelles, hypermnésie, droit à l'oubli, loi du 6 janvier 1978.
	
	Qu'est-ce que la RGPD ? Quel rôle joue la CNIL vis à vis de la RGPD ?


	\item[Thème 5 : Facebook, Wikipedia] 
	Facebook : Fonctionnement de l'application. Historique rapide des réseaux sociaux. Service rendu (et en échange de quoi ?) Dérives par rapport à la vie privée ? Problèmes et incidents provoqués par un mauvais usage de Facebook...

	Wikipedia : Fonctionnement du site,­ qui écrit ? ­ Qui contrôle l'information distribuée ? ­ Abus ? Fiabilité ?

	Mise en opposition et/ou rapprochements...


	\item[Thème 6 : Économie du numérique et Big Data]
	De nos jours, de grandes quantités de données sont envoyés à des entreprises.

	Quelles sont ces données ? Qu'en font les entreprises ? Comment gèrent-elles toute cette masse d'information ?


	\item[Thème 7 : Apprentissage automatique et intelligence artificielle]
	La quantité de données disponibles et surtout l'augmentation des capacités de traitement de ces données massives (big data) ont permis à l'apprentissage automatique de produire de très bons résultats dans différents domaines, notamment en utilisant des réseaux de neurones artificiels (apprentissage profond).
	
	Tous ces progrès modifient nos sociétés et doivent donc amener le citoyen à s'interroger sur leurs conséquences du point de vue éthique, social, politique et juridique.
	
	Chercher aussi l'origine de l'expression "Intelligence Artificielle".
	
	
	\item[Thème 8 : Informatique et environnement]
	La maintenance, la production et le développement des réseaux, des réseaux intelligents (smart grids), des nuages (clouds), des infrastructures de stockage, des supports numériques, mobiles ou non, se sont développés de façon considérable en quelques décennies.
	
	La création de centres de données induit par ailleurs des coûts écologiques conséquents qu'il s'agisse de consommation énergétique, de réchauffement climatique, de consommation des terres rares, de recyclage.
	
	Des démarches visant à limiter l'impact sur l'environnement existent telles que les filières légales de recyclage, d'autres sont à l'étude mais restent encore à développer. La bonne gestion de ces technologies peut être source de limitation des impacts voire de gains écologiques.
	
	
	\item[Thème 9 : Les algorithmes de décision et la question de la transparence]
	Certaines décisions sont aujourd'hui prises à l'aide d'algorithmes. Quels avantages cela présente-t-il ? Quels en sont les inconvénients ? Quelles exigences définir sur le plan de la transparence des processus ?
	
	
	\item[Thème 10 : Biométrie et sécurité informatique]
	La biométrie est souvent présentée comme l'outil ultime de sécurisation. Est-ce vrai ?
	
	Peut-on tromper une identification biométrique ? Ceci présente-t-il des risques pour l'usager ?
	
	Quels sont les risques de dérive à stocker des informations biométriques ?
	
	Comparer les systèmes classiques d'identification avec ceux utilisant la biométrie.
	
	
	\item[Thème 11 : Dématérialisation des démarches administratives et civiques]
	
	Y-a-t-il des avantages de la dématérialisation en terme écologique ? En terme de temps ?
	
	Analyser les problèmes engendrés comme celui de l'identification (pour les cartes d'identité ou les impôts), ou celui de la confiance (le vote électronique est dénoncé par beaucoup de personnes averties).
\end{description}
